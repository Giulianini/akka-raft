% -*- root: ../../main.tex -*-
\section{Dettagli Implementativi}
Qui di seguito verranno descritte in dettaglio alcune scelte implementative degne di nota che hanno permesso di migliorare la qualità del codice e l'integrazione fra i vari paradigmi di programmazione.

  \subsection{Incapsulamento di oggetti in attori}
  Utilizzando un approccio ad attori in un linguaggio OO e funzionale come Scala, è necessario introdurre nuovi pattern di programmazione al fine di garantire una buona integrazione. Nel caso di \textbf{classi modulari} come un interfaccia grafica, queste non possono essere trasformate semplicemente in un attore (estendendo la classe attore di Akka) ma devono essere \textbf{incapsulate in un attore esterno}. L'interazione fra modulo e attore avverrà tramite il \textbf{pattern observer} permettendo cosi di mantenere un'ottima separazione di ruoli.

  \subsection{Scala Self-Types per Disaccoppiamento Behaviours}
  Una annotazione self type di un trait definisce il tipo della classe che può estendere il trait. Una classe concreta che effettua il mixin di un trait deve assicurarsi quindi che il suo self-type (\texttt{this}) sia conforme a quello del trait con cui è mixata. Ciò significa che l'utilizzo di un self type permette il \textbf{mixing} dello stesso \textbf{trait} \textbf{all'interno della classe} anche se non lo estende direttamente; questo fa si che i \textbf{membri del trait} mixato possano essere \textbf{disponibili alla classe} senza dover utilizzare import.\\
  Nel nostro caso i self types sono stati utilizzati per disaccoppiare i vari behaviours \texttt{leader}, \texttt{candidate} e \texttt{follower} e renderli disponibili al \texttt{server} in modo separato. Un ulteriore self type è stato utilizzato per gestire la \texttt{Discovery} dei nodi del cluster; dato che la discovery consisteva in un'operazione esterna a RAFT si è deciso di ``nasconderla'' agli attori.

  \subsection{ScalaDoc}
  Ulteriori dettagli implementativi sono descritti e documentati in maniera approfondita sotto forma di \textbf{ScalaDoc}. La ScalaDoc viene generata automaticamente all'avvio della \textbf{build} di Gradle oppure usando il task scaladoc.
  \begin{lstlisting}[language=bash]
  ./gradlew scaladoc
  \end{lstlisting}
  E' possibile visitare la Scaladoc al link che segue: \href{http://akka-raft.surge.sh/}{Akka-Raft ScalaDoc}.