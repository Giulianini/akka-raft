% -*- root: ../../main.tex -*-               
\section{Analisi dei Requisiti}
L'applicativo è una rappresentazione operativa di RAFT; per questo motivo, i dettagli dei requisiti che deve avere sono insiti nelle caratteristiche dell'algoritmo discusse in precedenza. 

Il sistema dovrà consistere in un insieme di nodi che rappresentano due entità principali:
	\begin{itemize}
		\item \textbf{Un client:} si occupa di inoltrare le richieste ai server e ricevere gli esiti.
		\item \textbf{$n$ server:} accolgono le richieste del client, le eseguono e comunicano il risultato. Ognuna di queste entità, in un dato istante, ricopre uno e un solo ruolo tra \textit{Leader, Candidate} e \textit{Follower}
	\end{itemize}

Dal momento che queste entità devono essere \textbf{indipendenti} ed interagire tra loro, si è scelto di utilizzare il \textbf{paradigma ad attori}, che rende possibile agire ad un livello di astrazione molto elevato permettendo di facilitare la gestione delle \textbf{dinamiche di interazione}.


	\subsection{Il paradigma ad attori} \label{Actors}
	Si tratta di un paradigma nato dagli stessi principi che hanno ispirato la programmazione ad \textbf{oggetti}. Nello specifico una classe è caratterizzata da due concetti fondamentali:
	\begin{itemize}
		\item{\textbf{Stato:}} rappresenta l'insieme dei valori delle variabili di una classe in un dato istante. 
		\item{\textbf{Comportamento:}} espresso sotto forma di istruzioni contenute nei metodi.
	\end{itemize}

	Il paradigma ad attori riprende questi aspetti, arricchendoli con una caratteristica fondamentale: il \textbf{flusso di controllo}. 
	Per \textbf{attore} si intende quindi un' entità \textbf{indipendente}, \textbf{univocamente identificata}, caratterizzata dai seguenti elementi:
	\begin{itemize}
		\item \emph{Una \textbf{mailbox}, alla quale arrivano messaggi provenienti da altri attori.}
		\item \emph{Uno \textbf{stato}, non accessibile dall'esterno.}
		\item \emph{Un \textbf{comportamento} (behaviour), che definisce il modo in cui l'attore reagisce ai messaggi ricevuti.}
		\item \emph{Un \textbf{flusso di controllo} indipendente.}
	\end{itemize}

	Per rendere possibile la realizzazione di un sistema ad attori, sono necessarie tre primitive:
	\begin{itemize}
		\item \textbf{Send:} per inviare un messaggio ad un altro attore, compreso il mittente.
		\item \textbf{Create:} per creare un attore figlio.
		\item \textbf{Become:} per cambiare il comportamento dell'attore. Ogni comportamento è caratterizzato da una diversa reazione ai messaggi ricevuti.
	\end{itemize}

	Il paradigma ad attori garantisce inoltre le seguenti proprietà fondamentali: 
	\begin{itemize}
		\item{\textbf{Reactiveness:}} l'attore è reattivo poichè reagisce in maniera \textbf{individuale} ai messaggi ricevuti, dal momento che ha un proprio, unico, \textbf{event loop}. Bisogna però seguire una \textbf{golden rule:} \emph{Never block the event loop}.
		\item{\textbf{State encapsulation:}} lo stato dell'attore, descritto dai parametri che esso assume in un dato momento, è incapsulato all'interno dello stesso, \textbf{impedendo accessi dall'esterno}. 
		\item{\textbf{Macrostep semantics:}} si inizia ad eseguire una nuova operazione solo dopo aver concluso la precedente. Questo rende la gestione di ciascun messaggio \textbf{atomica}.
		\item{\textbf{Fairness:}} i messaggi accodati, \textbf{prima o poi} verranno gestiti. 
		\item{\textbf{Location transparency:}} gli attori possono comunicare tra di loro \textbf{senza} essere a conoscenza delle rispettive posizioni.
	\end{itemize}

	Esistono diverse implementazioni del paradigma ad attori, che differiscono per il tipo e la gestione della \textbf{receive}. I sistemi più moderni preferiscono un approccio \textit{a receive implicita}, dove la ricezione dei messaggi è gestita in maniera trasparente. Nelle sezioni successive si vedrà che, per quanto riguarda la realizzazione di questo progetto, la scelta è ricaduta proprio su questo tipo di implementazione.


