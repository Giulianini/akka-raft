% -*- root: ../../main.tex -*-               
\section{Analisi dei Requisiti} \label{Actors}
L'applicativo è una rappresentazione operativa di RAFT; per questo motivo, i dettagli dei requisiti che deve avere sono insiti nelle caratteristiche dell'algoritmo discusse in precedenza. 

Il sistema dovrà consistere in un insieme di nodi che rappresentano due entità principali:
	\begin{itemize}
		\item \textbf{Un client:} è l'entità che si occupa di inoltrare le richieste ai server e ricevere gli esiti.
		\item \textbf{$n$ server:} è l'entità che accoglie le richieste del client, le esegue e comunica il risultato. Ognuna di queste entità, in un dato istante, ricopre uno e un solo ruolo tra \textit{Leader, Candidate} e \textit{Follower}
	\end{itemize}

Dal momento che queste entità devono essere indipendenti ed interagire tra loro, si è scelto di utilizzare il paradigma ad attori, che rende possibile agire ad un livello di astrazione molto elevato permettendo di facilitare la gestione delle dinamiche di interazione.


	\subsection{Il paradigma ad attori}
	Un attore rappresenta una entità indipendente che ha le seguenti caraterristiche:
	\begin{itemize}
		\item \emph{possiede un proprio flusso di controllo}
		\item \emph{mantiene uno stato, non accessibile dall'esterno}
		\item \emph{può ricevere messaggi da altri attori e agire di conseguenza.}
		\item \emph{può inviare messaggi ad altri attori.}
	\end{itemize}
% -*- root: ../../main.tex -*-               
\section{Analisi dei Requisiti} \label{Actors}
L'applicativo è una rappresentazione operativa di RAFT; per questo motivo, i dettagli dei requisiti che deve avere sono insiti nelle caratteristiche dell'algoritmo discusse in precedenza. 

Il sistema dovrà consistere in un insieme di nodi che rappresentano due entità principali:
	\begin{itemize}
		\item \textbf{Un client:} è l'entità che si occupa di inoltrare le richieste ai server e ricevere gli esiti.
		\item \textbf{$n$ server:} è l'entità che accoglie le richieste del client, le esegue e comunica il risultato. Ognuna di queste entità, in un dato istante, ricopre uno e un solo ruolo tra \textit{Leader, Candidate} e \textit{Follower}
	\end{itemize}

Dal momento che queste entità devono essere indipendenti ed interagire tra loro, si è scelto di utilizzare il paradigma ad attori, che rende possibile agire ad un livello di astrazione molto elevato permettendo di facilitare la gestione delle dinamiche di interazione.


	\subsection{Il paradigma ad attori}
	Un attore rappresenta una entità indipendente che ha le seguenti caraterristiche:
	\begin{itemize}
		\item \emph{possiede un proprio flusso di controllo}
		\item \emph{mantiene uno stato, non accessibile dall'esterno}
		\item \emph{può ricevere messaggi da altri attori e agire di conseguenza.}
		\item \emph{può inviare messaggi ad altri attori.}
	\end{itemize}
