% -*- root: ../../main.tex -*-               
\section{Analisi dei Requisiti}
L'applicativo è una rappresentazione operativa di RAFT; per questo motivo, i dettagli dei requisiti che deve avere sono insiti nelle caratteristiche dell'algoritmo discusse in precedenza. 

Il sistema dovrà consistere in un insieme di nodi che rappresentano due entità principali:
	\begin{itemize}
		\item \textbf{Un client:} si occupa di inoltrare le richieste ai server e ricevere gli esiti.
		\item \textbf{$n$ server:} accolgono le richieste del client, le eseguono e comunicano il risultato. Ognuna di queste entità, in un dato istante, ricopre uno e un solo ruolo tra \textit{Leader, Candidate} e \textit{Follower}
	\end{itemize}

Dal momento che queste entità devono essere \textbf{indipendenti} ed interagire tra loro, si è scelto di utilizzare il \textbf{paradigma ad attori}, che rende possibile agire ad un livello di astrazione molto elevato permettendo di facilitare la gestione delle dinamiche di interazione.


	\subsection{Il paradigma ad attori} \label{Actors}
	Si tratta di un paradigma nato per riuscire a sfruttare al meglio le potenzialità della programmazione ad \textbf{oggetti} in ambito \textbf{concorrente}, interamente basato sul concetto di \textbf{attore}: \emph{everything is an actor}.

	Per \textbf{attore} si intende un' entità indipendente, univocamente identificata, caratterizzata dai seguenti elementi:
	\begin{itemize}
		\item \emph{Una mailbox, alla quale arrivano messaggi provenienti da altri attori.}
		\item \emph{Uno \textbf{stato}, non accessibile dall'esterno.}
		\item \emph{Un \textbf{comportamento} (behaviour), che definisce il modo in cui l'attore reagisce ai messaggi ricevuti.}
		\item \emph{Un \textbf{flusso di controllo} indipendente.}
	\end{itemize}

	Per rendere possibile la realizzazione di un sistema ad attori, sono necessarie tre primitive:
	\begin{itemize}
		\item \textbf{Send:} per inviare un messaggio ad un altro attore, compreso il mittente.
		\item \textbf{Create:} per creare un attore figlio.
		\item \textbf{Become:} per cambiare il comportamento dell'attore. Ogni comportamento è caratterizzato da una diversa reazione ai messaggi ricevuti.
	\end{itemize}

	

