% -*- root: ../../main.tex -*-
\section{Istruzioni per il deployment}
	Il progetto è stato scritto in \texttt{Scala} per la parte core, mentre la parte grafica è stata scritta con il supporto della libreria \texttt{JavaFX}.
	Per la gestione delle dipendenze, per la compilazione e il testing abbiamo optato per il tool \texttt{Gradle}.

	\subsection{Installazione dei requisiti}
	 Per compilare e eseguire l'applicativo è necessario munirsi di:
	 \begin{enumerate}
	 	\item \textbf{Java Development Kit:}
	 		Nonostante sià possibile utilizzare un qualunque \texttt{JDK} superiore alla versione 8, consigliamo di utilizzare l'\texttt{Oracle JDK-8}.
	 		Un altra valida alternativa è la distribuzione \texttt{Amazon Corretto-8}\cite{AmazonCorrettoSite}.
	 		Queste distribuzioni incorporano nativamente \texttt{JavaFX}, permettendo di risparmiare così ulteriori installazioni.
	 		
	 	\item \textbf{Scala: }
	 		Per l'esecuzione del programma è necessario munirsi di \texttt{Scala 2.12.10}, non garantiamo che un installazione più recente sia retrocompatibile.
	 	 
	 \end{enumerate}
 		
 	\subsection{Build e run}
	 	L'adozione di \texttt{Gradle} ha reso molto semplice la compilazione del progetto.
	 	Per la generazione di un singolo \texttt{jar} eseguibile completo, è stato scelto l'utilizzo del plugin \texttt{Gradle Shadow}\cite{GradleShadowSite} che permette la generazione di un jar asestante che soddisfi tutte le dipendenze internamente.\\
	 	Per compilare il progetto è sufficiente recarsi nella cartella principale e lanciare il seguente comando:
	 	\begin{itemize}
	 		\item Per i sistemi \textbf{Unix}: \texttt{./gradlew build} 
	 		\item Per i sistemi \textbf{Windows}: \texttt{gradlew.bat build} 
	 	\end{itemize}
		Il comando procederà alla compilazione dei sorgeti scala, fornendo in uscita un \texttt{jar} eseguibile che sarà collocato nella cartella /build/libs sotto il nome di \texttt{akka-raft-0.0.1-SNAPSHOT-all.jar}\\
		Il \texttt{jar} esegue internamente il main presente nell'\texttt{object JarLaunher} del pakage akka\_raft.
	  Per lanciare il jar è necessario eseguire il comando \texttt{scala tipologia id porta}.\\
		Dove le \textbf{tipologie} disponibili sono: client e server, la tipologia client esegue il \texttt{Client};
		mentre la tipologia server lancia un solo \texttt{Server}.
		Per l'esecuzione completa del cluster è necessario eseguire un solo client e esattamente cinque server, l'esecuzione di un numero diverso non permetterà al cluster di proseguire proseguire correttamente.\\
		L'\textbf{id} è una stringa identificativa dell'attore che viene eseguito, è possibile scegliere la stringa che più piace, a patto che sià univoca.\\
		Infine la \textbf{porta} è il numero di porta utilizzato da akka remoting per ogni attore, la porta zero equivale a fornire una porta random libera.\\
		Per l'esecuzione del cluster akka è necessario lanciare due seed node, i quali devono possedere una porta specifica(rispettivamente 5000 e 5001), definite nel file cluster.conf.\\
		Si consigia di eseguire sei istanze del jar seguendo il seguente ordine:
		  \begin{enumerate}
			  \item scala akka-raft-0.0.1-SNAPSHOT-all.jar client C0 5000
				\item scala akka-raft-0.0.1-SNAPSHOT-all.jar client S0 5001
				\item scala akka-raft-0.0.1-SNAPSHOT-all.jar client Sn 0
			\end{enumerate}	
		Sarà necessario eseguire il terzo comando con porta zero altre quattro volte, per raggiungere il numero esatto necessario per la partenza del cluster.
 
 \subsection{Test}
 Sono stati prodotti vari test, automatizzati e grafici; per il lancio dei test è stato creto un task chiamato \texttt{spec}, utilizzato per eseguire i test scala con \texttt{org.scalatest.tools.Runner}.\\
	Per l'esecuzione dei test è sufficente eseguire gradle tramite il comando:
	\begin{itemize}
		\item Per i sistemi \textbf{Unix}: \texttt{./gradlew test} 
		\item Per i sistemi \textbf{Windows}: \texttt{gradlew.bat test} 
	\end{itemize}
