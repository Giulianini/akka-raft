% -*- root: ../../main.tex -*-
\section{Istruzioni per il deployment}
	Il progetto è stato scritto in \texttt{Scala} per la parte core, mentre la parte grafica è stata scritta con il supporto della libreria \texttt{JavaFX}.
	Per la gestione delle dipendenze, per la compilazione e il testing abbiamo optato per il tool \texttt{Gradle}.

	\subsection{Installazione dei requisiti}
	 Per compilare e eseguire l'applicativo è necessario munirsi di:
	 \begin{enumerate}
	 	\item \textbf{Java Development Kit:}
	 		Nonostante sià possibile utilizzare un qualunque \texttt{JDK} superiore alla versione 8, consigliamo di utilizzare l'\texttt{Oracle JDK-8}.
	 		Un altra valida alternativa è la distribuzione \texttt{Amazon Corretto-8} \cite{AmazonCorrettoSite}.
	 		Queste distribuzioni incorporano nativamente \texttt{JavaFX}, permettendo di risparmiare così ulteriori installazioni.
	 		
	 	\item \textbf{Scala: }
	 		Per l'esecuzione del programma è necessario munirsi di \texttt{Scala 2.12.10}, non garantiamo che un installazione più recente sia retro-compa-tibile.
	 	 
	 \end{enumerate}
 		
 	\subsection{Build e run}
	 	L'adozione di \texttt{Gradle} ha reso molto semplice la compilazione del progetto.
	 	Per la generazione di un singolo \texttt{jar} eseguibile completo, è stato scelto l'utilizzo del plugin \texttt{Gradle Shadow} \cite{GradleShadowSite} che permette la generazione di un jar a sé stante che soddisfi tutte le dipendenze internamente.\\
	 	Per compilare il progetto è sufficiente recarsi nella cartella principale e lanciare il seguente comando:
	 	\begin{itemize}
	 		\item Per i sistemi \textbf{Unix}: 
	 			\begin{lstlisting}[language=bash]
./gradlew build
	 			\end{lstlisting}
	 		\item Per i sistemi \textbf{Windows}: 
	 			\begin{lstlisting}[language=bash]
gradlew.bat build
	 			\end{lstlisting} 
	 	\end{itemize}
		Il comando procederà alla compilazione dei sorgenti scala, fornendo in uscita un \texttt{jar} eseguibile che sarà collocato nella cartella \emph{/build/libs} sotto il nome di \texttt{akka-raft-1.0.0-all.jar}\\
		Il \texttt{jar} esegue internamente il main presente nell'\texttt{object JarLaunher} del pakage akka\_raft.
	  Per lanciare il jar è necessario eseguire il comando
    \[
      \texttt{scala akka-raft.jar tipologia id porta}
    \]
    \begin{itemize}
      \item{\textbf{Tipologia:}}
      Le \textbf{tipologie} disponibili sono: client e server, la tipologia client esegue il \texttt{Client};
      mentre la tipologia server lancia un solo \texttt{Server}.
      Per l'esecuzione completa del cluster è necessario eseguire un solo client e esattamente cinque server, l'esecuzione di un numero diverso non permetterà al cluster di proseguire proseguire correttamente.
      \item{\textbf{ID:}}
      L'\textbf{id} è una stringa identificativa dell'attore che viene eseguito, è possibile scegliere la stringa che più ci piace, a patto che sia univoca.
      \item{\textbf{Porta:}}
      Infine la \textbf{porta} è il numero di porta utilizzato da akka remoting per ogni attore, la porta zero equivale a fornire una porta random libera.
    \end{itemize}
		Per l'esecuzione del cluster Akka è necessario lanciare due \textbf{seed node}, i quali devono possedere una porta specifica(rispettivamente 5000 e 5001), definite nel file \texttt{cluster.conf}.\\
		Si consiglia di eseguire \textbf{sei istanze} del jar seguendo il seguente ordine:
		  \begin{enumerate}
			  \item
			  	\begin{lstlisting}[language=bash]
scala akka-raft-1.0.0-all.jar client C0 5000
			  	\end{lstlisting}
			  	\begin{lstlisting}[language=bash]
scala akka-raft-1.0.0-all.jar client S0 5001
					\end{lstlisting}
			  	\begin{lstlisting}[language=bash]
scala akka-raft-1.0.0-all.jar client Sn 0
					\end{lstlisting}			  
			\end{enumerate}	
		Sarà necessario eseguire il terzo comando con porta $0$ altre quattro volte, per raggiungere il numero esatto necessario per la partenza del cluster.
 
 \subsection{Test}
 Sono stati prodotti vari test, automatizzati e grafici; per il lancio dei test è stato creato un task chiamato \texttt{spec}, utilizzato per eseguire i test scala con \texttt{org.scalatest.tools.Runner}.\\
	Per l'esecuzione dei test è sufficiente eseguire gradle dalla cartella principale del progetto, tramite il lancio del comando:
	\begin{itemize}
		\item Per i sistemi \textbf{Unix}: 
			\begin{lstlisting}[language=bash]
./gradlew test
			\end{lstlisting}
		\item Per i sistemi \textbf{Windows}: 
			\begin{lstlisting}[language=bash]
gradlew.bat test
			\end{lstlisting}
	\end{itemize}
