% -*- root: ../main.tex -*-
\chapter{Conclusioni}
Giunti alle conclusioni di questo progetto possiamo affermare di ritenerci estremamente soddisfatti del lavoro svolto. Il risultato ottenuto ha superato di gran lunga le nostre aspettative ma la strada che ha portato alla sua realizzazione è stata spesso impervia.\\
La realizzazione di un algoritmo di consenso richiede un enorme mole di studio e lavoro. Nel nostro caso il lavoro svolto si è suddiviso in questo modo: 
\begin{itemize}
  \item{\emph{Studio e documentazione:}}
  abbiamo impiegato all'incirca una settimana di lavoro al fine di comprendere e documentare al meglio l'algoritmo RAFT. Una volta che l'algoritmo è stato compreso si è potuto lavorare sull'effettiva implementazione concentrandosi inizialmente sulle varie tecnologie implementative. Dopo una breve indagine è stato scelto il framework Akka.
  \item{\emph{Studio di Scala:}}
  una volta che è stato individuato il framework di sviluppo, la scelta del linguaggio è stata quasi obbligata. Akka infatti, come abbiamo già visto, in Scala, permette di lavorare in modo estremamente conciso ed efficiente. L'apprendimento di Scala non è stato immediato e per comprenderlo al meglio abbiamo impiegato una settimana di studio full-time.
  \item{\emph{Studio di Akka:}}
  appreso il linguaggio Scala ci siamo concentrati sullo studio del framework Akka, in particolare ci siamo concentrati sulla libreria di Akka dedicata agli attori e al clustering.
  \item{\emph{Implementazione vera e propria:}}
  finalmente appresi i linguaggi e le tecnologie necessarie ci siamo concentrati sull'implementazione dell'algoritmo RAFT. Siamo partiti dal modello e abbiamo proseguito a sviluppare in modo incrementale i vari attori.
  \item{\emph{Debugging:}}
  completata l'implementazione l'algoritmo ha presentato un numero considerevole di bug. La fase di debugging è stata la fase più frustrante del progetto ma dopo grandi difficoltà siamo riusciti ad ottenere un'implementazione funzionante e pulita dell'algoritmo.
\end{itemize}


\section{Lavori Futuri}
  Il lavoro svolto come abbiamo visto è limitato a una semplice implementazione dell'algoritmo RAFT. Dato che l'algoritmo è stato sviluppato relativamente a un \textbf{caso d'uso specifico}, \textit{conti correnti bancari}, esso manca di estendibilità e genericità. Uno sviluppo futuro potrebbe risiedere nella possibilità di fornire una \textbf{versione generica} dell'algoritmo che sarà poi estendibile concretamente in base al caso d'uso.\\
  Un'altra possibile estensione futura dell'implementazione consiste nell'inserimento delle \textit{features extra} fornite dall'algoritmo RAFT: \textbf{Log Compaction} e \textbf{Configuration Changes}.

\section{Cosa Abbiamo Imparato}
  Quando si sceglie un progetto sostanzioso spesso si rischia di rimanere sommersi dal carico di conoscenze necessarie per portare a termine il lavoro. Nel nostro caso ci sono stati momenti di sconforto che, con supporto reciproco, siamo riusciti a superare. Gli argomenti che inizialmente si sono mostrati più ostici alla fine sono entrati a far parte del nostro bagaglio conoscitivo.
  \begin{itemize}
    \item{\emph{Scala:}}
    Avendo a che fare con il linguaggio Scala la sua comprensione è avvenuta in modo passivo, inoltre essa ha portato con se interessanti conseguenze. Siamo entrati in contatto con un nuovo paradigma di programmazione quello \textbf{funzionale} e abbiamo imparato a lavorare in un ambiente OOP puro.
    \item{\emph{Akka:}}
    Il framework Akka è sterminato e uno studio di un mese non basterebbe a coprire tutte le estensioni presenti. Nel nostro caso ci siamo concentrati solo sui moduli dedicati agli attori e al clustering e possiamo affermare che la conoscenza dei moduli è completa.
    \item{\emph{Algoritmi di Consenso:}}
    Studiando RAFT abbiamo compreso il \textbf{funzionamento} e la \textbf{complessità} di un \textbf{algoritmo di consenso}, inoltre abbiamo toccato con mano la \textbf{complessità insita nelle interazioni fra entità nel distribuito}. Infine lo studio di RAFT ci ha permesso di entrare a conoscenza del fantastico mondo delle \textbf{macchine a stati replicate}.
  \end{itemize}


\section*{Note Stilistiche}
\begin{itemize}
  \item Le entità di progetto sono espresse in formato \texttt{Teletypefont}.
  \item Gli indici e le variabili numeriche è stata usata la modalità \textit{math}. Es $n \in C$
  \item Gli spezzoni di condice sono espressi in formato \textit{listing}
  \begin{lstlisting}[language=bash]
echo ``Hello World''
  \end{lstlisting}
\end{itemize}